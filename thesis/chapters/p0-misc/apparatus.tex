%!TEX TS-program = xelatex
%!TEX root = ../../maxwell2018thesis.tex

\begin{preamble}[Apparatus Used]
\phantomsection
\addcontentsline{toc}{part}{Apparatus Used}

The extensive simulations of interaction reported in this thesis were run on two computers hosted by the School of Computing Science at the University of Glasgow. Basic hardware and software specifications are listed below. FQDNs are obscured to avoid potential security issues.

\begin{itemize}
    
    \item{\texttt{t*****.***.gla.ac.uk}\\\emph{8x} \emph{AMD Opteron}\texttrademark~processors 6366 HE, 64 logical cores\\512GB RAM\\Fedora 18 \emph{(Spherical Cow)}, \texttt{3.11.10-100.fc18.x86\_64}}
    
    \item{\texttt{f*****.***.gla.ac.uk}\\\emph{2x} \emph{Intel}\textregistered~\emph{Xeon}\textregistered~CPU E5-2660, 32 logical cores\\128GB RAM\\Scientific Linux 6.10 \emph{(Carbon)}, \texttt{2.6.32-573.12.1.el6.x86\_64}}
    
\end{itemize}

Both computers ran experiments with \emph{Python} 2.7.14, \emph{Whoosh} 2.7.4 and the \simiir~framework. Our thanks go to Douglas Macfarlane, Stewart MacNeill and the rest of the support team at the School for ensuring these computers were available for the experimentation work.
\end{preamble}