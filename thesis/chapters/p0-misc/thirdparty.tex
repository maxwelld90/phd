%!TEX TS-program = xelatex
%!TEX root = ../../maxwell2019thesis-bcs.tex

\begin{preamble}[Third-Party Content]

Illustrations are used extensively throughout this thesis to make reading it a little more enjoyable, as well as (hopefully) providing the reader with a better understanding of the points and concepts being conveyed. Most of the illustrations were drawn by the author of this thesis in \emph{Adobe \textregistered~Illustrator \textregistered~CS6}.

However, several vector artworks have also been downloaded from \texttt{freepik.com} and incorporated within illustrations. This statement serves as an acknowledgement that such artworks have been incorporated within this thesis, and are included on the assumption that \textit{no part of this work will be used for commercial purposes.} Acquired licences that permit the author to use these illustrations are available on request.

This \textit{IKEA} assembly man is used at the start of Part~\ref{part:stopping} to convey the idea of assembling the searcher model proposed in this thesis. \emph{Inter IKEA Holding S.A.} granted approval to incorporate him within this work. \emph{Thank you, IKEA!}

The required licences to publish this thesis in PDF format using \headerfont\selectfont Foundry Sterling\normalfont\selectfont~have been procured by the author from \texttt{fonts.com}.

Finally, Picture~\ref{fig:acks_friends} in the \genericblack{PhD Journey} shows several of my friends from the School of Computing Science at the University of Glasgow. I sought permission from everyone sitting at the table before including the image in my thesis. \emph{Thanks, team!}

For a detailed list of all third-party illustrations used, refer to the accompanying website for this thesis at \texttt{https://www.dmax.org.uk/thesis}.

\end{preamble}

\newpage