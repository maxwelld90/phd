%!TEX TS-program = xelatex
%!TEX root = ../../maxwell2018thesis.tex

\begin{preamble}[Thesis Abstract]
\phantomsection
\addcontentsline{toc}{part}{Thesis Abstract}

Stopping is a fundamental aspect of animal -- and by definition, human -- behaviour and decision making.
Example or two of stopping examples.
Similarly, when seeking information, a searcher must make a decision as to when he or she should top examining.
Given the complexities of the IIR process, stopping during search can occur at a number of different points.
Ranked list of results.
Session.
Stopping has been until recently an ill-examined phenonomemon, perhaps because it is an inherently difficult task to model.

In this thesis, we consider stopping from a modelling perspective.
Given stopping heuristics defined in the literature, we turn them into a number of different operationalisable stopping strategies.


% \emph{Interactive Information Retrieval (IIR)} is a complex, non-trivial process~\citep{ingwersen2005theturn}. During the course of a search session, a searcher may issue multiple queries, and for each query, examine a varying number of result summaries (textual snippets) and documents. That is to say, the interactions and behaviours exhibited by a searcher are \emph{complex}, and he or she will inherently adapt their behaviour based upon a number of factors -- perhaps most notably due to the perceived relevance of the list of results presented to them~\citep{moffat2013users_versus_models}. A searcher, for example, would be wise to abandon a list of results if they are perceived to be of low quality.
%
% Despite the findings of a large number of studies in the field of IIR, many of the models and measures still in use today within the \emph{Information Retrieval (IR)} community do not consider these complex factors regarding user behaviours. Central to these models and measures is the so-called \emph{Cranfield Paradigm}, the \emph{de facto} model used for IR evaluation, developed in the early 1960's. While this approach has been adapted over the years to suit ever more complex evaluation tasks~\citep{harman2010cranfield}, the model is still largely systems-focused, rather than user-focused. Such an evaluation approach considers a single query, and the consideration of a large number of documents returned from that query to be relevant -- a wholly unrealistic approach.
%
% As such, many researchers have proposed different models and frameworks for the purposes of IIR modelling and evaluation. In this thesis, we propose the \emph{Complex Searcher Model (CSM)}, a high-level model that attempts to capture the various complex interactions and decisions that take place during \emph{ad-hoc topic retrieval}. Central to the CSM is the inclusion of several key \emph{decision points}, which provide a means for modelling the \emph{stopping behaviour} of a searcher. Stopping behaviour is a fundamental aspect of human behaviour~\citep{nickles1995judgment}, and during search, this is no different. Research into the stopping behaviours has however until recently been sparse, with work suggesting that searchers stop when they feel that what they have found is simply ``good enough'' to stop~\citep{wu2014information_scent}.
%
% Considering the CSM, this thesis attempts to ascertain a more precise definition of what is ``good enough''. We operationalise a variety of different stopping \emph{heuristics} defined within the literature over a number of years, considering heuristics from both IR research and those based upon ecology, where researchers examined the stopping behaviour behaviour of animals when foraging for food. These operationalised heuristics are examined in varying search contexts. We compare the behaviour of real-world subjects who partook in a series of user studies against simulated behaviours utilising the CSM.
%
% From the work undertaken as part of this thesis, we are able to show that developing searcher models incorporating some form of stopping behaviour is able to offer more realistic, and credible simulations of the complex interactions that take place during a search session.
\end{preamble}