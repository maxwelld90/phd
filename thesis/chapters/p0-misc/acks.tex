%!TEX TS-program = xelatex
%!TEX root = ../../maxwell2018thesis.tex

\begin{tikzpicture}[remember picture, overlay]
      \node[anchor=north west] at (-2.88,3.475){%
        \scalebox{0.66}{\pgfimage{figures/ch0-acks_header.pdf}}};
\end{tikzpicture}

\begin{preamble}[The PhD Journey]
\setcounter{footnote}{0}
\phantomsection
\addcontentsline{toc}{part}{The PhD Journey}

A good friend of mine (and a fellow PhD) student once said to me that when the time came to write his PhD thesis, he would avoid an acknowledgements section where \emph{everyone and their} dog would be thanked for helping him reach his target of attaining a PhD. I on the other hand take a very different light on this matter. There are a lot of people who have in one way or another helped me reach where I am today. Whether these people actively guided me in my studies, or were individuals who I was fortunate to become acquainted with over the past five years, they all \emph{``cajoled''}\footnote{Professor Ian Ruthven used this term in his PhD thesis~\citep{ruthven2001phd} as a means of describing the individuals who were there for him, behind the scenes, \emph{``cajoling''} him towards the finishing line.} me in one way or another towards the finishing line.

I firmly believe that everybody who I have the pleasure of meeting and working with over the past five years should be acknowledged -- whether they feel they contributed in any meaningful way. If you are one of these people and are left wondering, believe me: \emph{you did make a difference.} While acknowledging everyone here is too little recompense for the opportunities and good times I have had over the past five years, I want to at least say thank you. \textbf{\emph{To show my sincere appreciation, I want to dedicate this work to each and every one of them}} -- regardless of whether they have a dog or not.

Hindsight tells me that doing a PhD is much like embarking on a \emph{very long,} solo journey. Unless you do one yourself, you won't appreciate how tough (and lonely) it can be at times. Three years in, I found myself sitting in my lab all alone on a Friday night, wondering why my experiments weren't producing the results I had expected and hoped for.\footnote{It was a stupid mistake, of course. From memory, I think I forgot to increment a counter in a loop. But it took an entire evening to figure that out. \emph{Of course it did!}} It can at times all seem so very pointless, and you find yourself questioning what you're doing with your life. I experienced these lows more times than I care to admit. It can be tortuous. \emph{Impostor syndrome} is something every PhD student feels, and I was no exception.

However, I got to the finishing line. Doing a PhD isn't just about learning your field of study and making an original contribution to it; no, it's much more than that. It also involves learning about yourself. It's \emph{character building.} It involves steely grit and determination to get through the difficult times. Even when everything comes crashing down around you, \emph{you will get through it.} My PhD taught me this more than anything, and for that I am incredibly thankful. I'm definitely a different person for having done it\footnote{This is something most people will agree with. My friend James acknowledged this in the acknowledgements of his excellent PhD thesis, too~\citep{mcminn2018phd}.} -- a much better one (I think so, anyway!), equipped with a good skillset to enter the world and make a positive contribution. Even though every PhD comes complete with negative moments, I took positives from all of them. From this, I could enjoy the good times even more. And believe me, there were \emph{heaps} of good times during the past five years.

One of the many great things about my experience as a PhD student was the office I was given to work in. It's in the \emph{Sir Alwyn Williams Building,} room 221. Being a contemporary building, there's lots of windows -- and you get a really nice view of the grass outside on Lilybank Gardens, and, yeah, the \emph{Boyd Orr Building,} too. However, in moments of reflection, I always found myself looking out the window at people walking past outside. I find it fascinating that everyone has their own story to tell. Everyone's experiences -- from all walks of life -- are different, making for a virtually limitless number of stories to listen to -- and to learn from. I have always found this truth about life to be absolutely fascinating.

So, on that basis, I want to spend the next few pages of the \emph{story of my PhD journey,} acknowledging everyone who made a positive impact along the way. I think that investing this additional time in writing this short passage is a good reflective experience, and also goes a little to recompensing for the amazing things these people have done for me.

I hope you enjoy reading it as much as I enjoyed writing it!

\acksep

\blueboxheader{Glasgow and SAWB221}
Glasgow uni folk who made life bearable.

\blueboxbold{Broadening my Horizons}
Where did I go? Who did I meet?

\blueboxbold{The Highs and Lows}
Good and bad times. More good than bad. But the bad are more character building, I think.

\blueboxbold{The Final Showdown}
Final exam. Iadh. Suzan. Michele.

\blueboxbold{Closer to Home}
Family.

\blueboxbold{Dr Azzopardi}
Bit for Leif

\acksep

Bit of a gap, then the spiel at the end.

\end{preamble}