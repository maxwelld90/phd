%!TEX TS-program = xelatex
%!TEX root = ../../maxwell2018thesis.tex

\begin{tikzpicture}[remember picture, overlay]
      \node[anchor=north west] at (-2.88,3.475){%
        \scalebox{0.66}{\pgfimage{figures/ch0-acks_header.pdf}}};
\end{tikzpicture}

\begin{preamble}[The PhD Journey]
\setcounter{footnote}{0}
\phantomsection
\addcontentsline{toc}{part}{The PhD Journey}

A good friend of mine (and a fellow PhD student) once said to me that when the time came to write his PhD thesis, he would avoid an acknowledgements section where \emph{everyone and their} dog would be thanked for helping him reach his target of attaining a PhD. I on the other hand take a very different light on this matter. There are a lot of people who have in one way or another helped me reach where I am today. Whether these people actively guided me in my studies, or were individuals who I was fortunate to become acquainted with over the past five years, they all \emph{``cajoled''}\footnote{Professor Ian Ruthven used this term in his PhD thesis~\citep{ruthven2001phd} as a means of describing the individuals who were there for him, behind the scenes, \emph{``cajoling''} him towards the finishing line.} me in one way or another towards the finishing line.

I firmly believe that everybody who I have the pleasure of meeting and working with over the past five years should be acknowledged -- whether they feel they contributed in any meaningful way. If you are one of these people and are left wondering, believe me: \emph{you did make a difference.} While acknowledging everyone here is too little recompense for the opportunities and good times I have had over the past five years, I want to at least say thank you. \textbf{\emph{To show my sincere appreciation, I want to dedicate this work to each and every one of them}} -- regardless of whether they have a dog or not.

Hindsight tells me that doing a PhD is much like embarking on a \emph{very long,} solo journey. Unless you do one yourself, you won't appreciate how tough (and lonely) it can be at times. Three years in, I found myself sitting in my lab all alone on a Friday night, wondering why my experiments weren't producing the results I had expected and hoped for.\footnote{It was a stupid mistake, of course. From memory, I think I forgot to increment a counter in a loop. But it took an entire evening to figure that out. \emph{Of course it did!}} It can at times all seem so very pointless, and you find yourself questioning what you're doing with your life. I experienced these lows more times than I care to admit. It can be tortuous. \emph{Impostor syndrome} is something every PhD student feels, and I was no exception.

However, I got to the finishing line. Doing a PhD isn't just about learning your field of study and making an original contribution to it; no, it's much more than that. It also involves learning about yourself. It's \emph{character building.} It involves steely grit and determination to get through the difficult times. Even when everything comes crashing down around you, \emph{you will get through it.} My PhD taught me this more than anything, and for that I am incredibly thankful. I'm definitely a different person for having done it\footnote{This is something most people will agree with. My friend James gave a nod to this in the acknowledgements of his excellent PhD thesis, too~\citep{mcminn2018phd}.} -- a much better one (I think so, anyway!), equipped with a good skillset to enter the world and make a positive contribution. Even though every PhD comes complete with negative moments, I took positives from all of them. From this, I could enjoy the good times even more. And believe me, there were \emph{heaps} of good times during the past five years.

One of the many great things about my experience as a PhD student was the office I was given to work in. It's in the \emph{Sir Alwyn Williams Building,} room 221. Being a contemporary building, there's lots of windows -- and you get a really nice view of the grass outside on Lilybank Gardens, and, yeah, the \emph{Boyd Orr Building,} too. However, in moments of reflection, I always found myself staring vacantly out the window at people walking past outside, going about their lives. Everyone's experiences -- from all walks of life -- are different, making for a virtually limitless number of stories to listen to -- and to learn from. I have always found this truth about life to be absolutely fascinating.

So, on that basis, I want to spend the next few pages writing about \emph{my PhD journey,} acknowledging everyone who made a positive impact along the way. I think that investing the additional time in writing this short passage is a good reflective experience, and also goes a little to recompensing for the amazing things these people have done for me.

I hope you enjoy reading it as much as I enjoyed writing it.

%\acksep

\blueboxheader{Day One \textemdash~and Looking Back}
Day one was October 1\textsuperscript{st}, 2013. I remember this day well. In particular, I have vivid memories of sitting down at my new desk in the morning and thinking something along the lines of \emph{``what have I just let myself in for?!''}. The very idea that I was now a PhD student in itself felt really daunting, because from reading research papers in my MSci year I was humbled by how much knowledge there was out there -- and I had to get myself to a level to contribute to that knowledge. After being assigned my first task by my supervisor, Leif, I set to work -- but it did seem very overwhelming.

However, I chipped away at it. As one task was completed, the next one fell into place -- and I found I could do that, too (with some guidance, of course!). I started to produce things, got a paper accepted after six months, and presented it at a conference (as I'll talk about shortly). But as I worked away, I started to find another area of research\footnote{User modelling and simulation -- the scope of this thesis.} to be much more interesting. The work presented in the thesis you are reading is actually pretty different from what I thought about doing back on day one. It just goes to show how when you think you have something laid out before you, it's by no means certain that it'll happen.

\blueboxheader{Life in Glasgow}
One constant that was present throughout my time as a PhD student at Glasgow was the people. There were always individuals I could rely upon for support, advice, or a simple chat. We'd often find ourselves down at \emph{Brel} when the sunshine was out (which did happen \emph{sometimes}). These are the people that I'd like to acknowledge first -- and what better than to start with those who I shared an office with over the past five years?

\blueboxheader{Team IR!}
To Stuart Mackie, {\arabicfont الصافوري ابراهيم امين فاطمه} (Fatma Elsafoury), my \emph{tocayo} Jorge David Gonz\'{a}lez Paule and {\asianfont 王烯} (Xi Wang) -- thank you for your companionship throughout the years in \emph{SAWB221.} The camaraderie and support we gave one another did not go unnoticed, and I am grateful for that. Fatma, thank you for the support and interesting discussions that we had. There's also two other individuals with whom I also shared \emph{SAWB221} with -- and also a home (for four years!). To Hora\c{t}iu Bota, thank you for the friendship that we had over the years throughout our time as PhD students. To {\thaifont \Large จรณะ มโนธรรมรักษา} (Jarana Manotumruksa), thank you for your friendship throughout. It's been an absolute pleasure, and it didn't feel like being in an office -- you all made it a happy place.

I'd also like to say thank you to my friends Colin McLellan and Andrew (James) McMinn. All three of us started at the \emph{University of Glasgow} back in September 2008 as undergraduates in Computing Science. Colin was in my very first \emph{CS1Q} undergraduate lab! By early 2019, all three of us had passed our PhD defences at the same institution, although the routes we took to get to that point were slightly different. \emph{We got through it together!} To Colin in particular, thank you for the support and friendship -- especially when we were both writing up at the end. Having the same supervisor kept us in close contact with one another -- but I don't think either of us have a bad word to say about Leif!

With all of us working on some aspect of the field \emph{Information Retrieval,} there's also a lot more people within the wider IR group that I would like to acknowledge. My appreciation goes out to everyone who resided next door in \emph{SAWB220} over the years, including {\arabicfont  الخوالدة سليمان رامي} (Rami Alkhawaldeh), {\arabicfont الدبعي عبدالرقيب شوقي} (Shawki Al-Dubaee), {\arabicfont العشبان ابراهيم نجود} (Nujud Aloshban), Phil McParlane, Jes\'{u}s Alberto Rodriguez P\'{e}rez (and his brother, F\'{e}lix Rodr\'{i}guez P\'{e}rez), Stewart Whiting, {\asianfont 辛鑫} (Xin Xin) and {\asianfont 发杰原} (Fajie Yuan). To Stewart in particular, thank you for your support throughout your time as a PhD student -- your guidance was greatly appreciated and valued when I started out. You made things seem a little less daunting.

I'd also like to acknowledge my friends in the rest of the IR group at Glasgow. In particular, I would like to acknowledge Jeff Dalton, Craig Macdonald, Richard McCreadie, Graham Mcdonald, {\asianfont 方安杰} (Anjie Fang) and {\asianfont 苏亭} (Ting Su) for their friendship and support throughout the years. Professor Iadh Ounis was also a great source of support throughout my time at Glasgow. Together with Leif and Craig, Iadh taught me many of the basics of IR in my MSci year, for which I am very grateful. Iadh was also one of the examiners for my final PhD defence -- and I'll talk about that experience later.

I'd also like to pay particular thanks to {\farsifont مشفقى ياشار} (Yashar Moshfeghi) for his friendship and support throughout my time at Glasgow. When you and Guido Zuccon were both PhD students at Glasgow, you were my tutors for the undergraduate \emph{Java Programming 2} course -- and you both helped reinforce many of the programming constructs that I use today! Yashar, your expert knowledge and advice on how to run crowdsourced studies was also appreciated. You played an important role in helping me to get the studies that I define and report on in this thesis up and running. Thank you.

\blueboxheader{Glasgow Computing Science}
Of course, I didn't just exclusively interact and socialise with people who studied IR. One of the great things about the \emph{School of Computing Science} at the University of Glasgow is its size, and the huge range of different disciplines that are studied. I have made friends with many people along the way, and also learnt things from different research areas, too! It's always interesting to see what other people are working on.

I made some close friends. To G\"{o}zel Shakeri, you are the best. I cannot thank you enough for your friendship, support and encouragement that you've given me throughout my time as a PhD student. The support and words of advice through the difficult times -- especially when my world came crashing down in mid-2018 -- will not be forgotten. Even if I was able to even begin offering you the advice and comfort that you did for me, I will have been a good friend to you, too. Thank you. And to Frances Cooper, thanks for your friendship and company, especially during the final write-up phase! In addition to G\"{o}zel and Frances, there's heaps of other people at Glasgow that I want to acknowledge. To name a few... Blair Archibald, Ornela Dardha, Marco Cook, Richard Cziv\'{a}, Euan Freeman, Simon Jouet, Φωτεινή Κατσαρού (Foteini Katsarou), William Kavanagh, Antoine Loriette, Ciaran McCreesh, Stephen McQuistin, Magnus Morton, Алекс Панчева (Alex Pancheva), Craig Reilly, Stefan Raue, Giorgio Roffo, Charlie Rutherford, Kyle Simpson, Robbie Simpson, Michel Steuwer, Lovisa Sundin, Patrizia Di Campli San Vito, Tom Wallis, David White and Михаил Янев (Mihail Yanev) over the years provided a friendly face and support. My appreciation goes out to every single one of you. Even if we simply had a drink, your company meant (and still means!) a lot.

I would also like to pay particular thanks to Professor Roderick Murray-Smith. Thank you for your support when Leif left Glasgow in mid-2016 to the \emph{University of Strathclyde.} Your insightful advice and feedback on my work throughout my time as a PhD student gave me an alternative perspective. From this, I was able to incorporate some of your points into the final product, hopefully making it a stronger thesis.

And to those friends I have stayed in touch with from my undergraduate days, I want to acknowledge your support and continued friendship. In particular, I'd like to acknowledge Julie Briand, Gary Christie, Ad\'{e}la Holubov\'{a} and Shaun Rew. Julie, thanks for your company throughout the process -- we both achieved our goals and got our PhDs! \emph{Choose your future. Choose life.} I'd also like to mention Sean McKeown -- thank you for your friendship throughout the whole experience. I value your advice and feedback, and I hope I have been able to repay that over the years. You have done \emph{Edinburgh Napier} proud.

\blueboxheader{Tutoring, Exam Collection and More}
I always said to my friends that when my PhD work was getting tough, I could find some solace in teaching. Throughout my time as a student in the School of Computing Science, I've been incredibly fortunate to take on such important roles -- and from those roles, meet and work with some fantastic people. Back when I was a fourth year undergraduate, Professor Quintin Cutts introduced me to the world of teaching. From that moment, I never looked back. Tutoring, demonstrating, and teaching was one of the best things I did at Glasgow. Sitting down and helping someone understand a solution for a problem that they have been facing in their work is such an enjoyable experience.

I tutored labs for a total of \emph{nine years} -- and loved every minute of doing so. While I tutored basics such as \emph{CS1P} and \emph{JP2} (Python and Java programming), my main focus was undoubtedly \emph{web development.} As I'll talk about more later, I wrote a book with Leif called \emph{Tango with Django} to make learning the \emph{Django} web application framework a more straightforward experience. I'd like to thank Professor David Manlove and Gerardo Arag\'{o}n-Camarasa for providing me with the opportunity to continue working on web development with them in my capacity as a tutor. I thoroughly enjoyed working with you both. And to my friend and fellow PhD student Laura Voinea, thank you for your company during the \emph{WAD2} and \emph{ITECH} labs over the years. Working with you was an absolute pleasure.

Of course, there's also the administrative team within the School that kept things flowing smoothly throughout my time here. These were the individuals who provided me with support when I needed it, too -- and I want to acknowledge them here. To Helen McNee and Αναστασία Φλιάτουρα (Anastasia Fliatoura), thank you both for making the PhD experience as straightforward as it could have been, at least from and administrative point of view! In particular, I want to thank Anastasia for her help in sorting out the thesis submission dates for me at the end of the PhD.

I also want to acknowledge Teresa Bonner, Helen Border and Gail Reat in the teaching office. You all trusted me to do the job that I did when it came to tutoring, and for that I am very grateful. One of the other jobs you gave me during my time as a PhD student was to run around the campus during exam season and collect the student's scripts. Although to many this sounds like a nightmare, I actually really enjoyed it. Once again, it provided a nice break from my studies, and I learnt how to sort \textasciitilde 200 exam scripts -- by matriculation number -- in the quickest possibile time. \emph{Where else would I have got that experience?} Thanks also to Magnus and Laura -- as well as Paul Harvey -- for your companionship when we spent those days in April-May 2015, 2016 and 2017 running around collecting all the student's scripts!

Of course, all of these extra commitments I took onboard were for the benefit of the students who have studied at the School over the years. As their tutor or demonstrator, I've had the good fortune to get to know some wonderful people over the years. Even if I guided them through their studies for a few weeks of their lives, I hope that I left an impact.

In particular, I want to acknowledge Екатерина Александрова (Ekaterina Aleksandrova), Lisa Brooks, {\asianfont 陳文勝} (Winston Chen), Άγγελος Κωνσταντινίδης (Angelos Constantinides), David Creigh, Tom Decke, Ивелина Дойнова (Ivelina Doynova), Leisha Hussein, Lisa Laux, Elena Lucchetti, Rebecca Orth, Gabriele Rossi, Vincent Schlatt and Tevhide Turkmen for keeping in touch and your friendship throughout our time here at Glasgow. It has been a pleasure. Sorry that you were inflicted with the pain of having to \emph{Tango with Django} -- but I know that in the end, you all tangoed really well!

\blueboxbold{Broadening my Horizons}
Where did I go? Who did I meet?

\blueboxbold{The Highs and Lows}
Good and bad times. More good than bad. But the bad are more character building, I think.

\blueboxbold{The Final Showdown}
Final exam. Iadh. Suzan. Michele.

\blueboxbold{Closer to Home}
Family.

\blueboxbold{Dr Azzopardi}
who have i been mentioning all of this time?
i want to mention that you left me alone at the end (sort of), and I'm so grateful for that, too.

\acksep

Bit of a gap, then the spiel at the end.

1962 days from commencement to passing the viva.

ending phase.- gabrielle said when I was leaving canberra that people move on.
i'm sure people will move on from glasgow. and everyone is getting on with their lives.
but the memories i have made with these people over the past five years are ones that i will take with me for whatever opportunities lie before me.

and you know what?

this is only the beginning!

\end{preamble}