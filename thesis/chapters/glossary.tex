%!TEX TS-program = xelatex
%!TEX root = ../maxwell2018thesis.tex

% To refer to a glossary entry, use the \gls{} command.
% If there is an acronym for the glossary entry, refer to the acronym, which in turn cross-references the glossary entry.

% To add a new glossary entry, use the following template.
%
% \newglossaryentry{glos:NAME}{
%     name={GLOSSARY ENTRY},
%     description={DESCRIPTION},
%     first={\emph{FIRST MENTION}},
% }
%
% Note that first won't be used when the acronym is used.


\newglossaryentry{glos:ir}{
    name={IR},
    description={The study of Information Retrieval},
}

\newglossaryentry{glos:iir}{
    name={IIR},
    description={The study of Interactive Information Retrieval},
}

\newglossaryentry{glos:www}{
    name={WWW},
    description={An information space in which documents and other resources, linked together via hypertext links, can be accessed via the Internet.},
}

\newglossaryentry{glos:stopping_heuristic}{
    name={Stopping Heuristic},
    plural={stopping heuristics},
    first={\emph{stopping heuristic}},
    firstplural={\emph{stopping heuristics}},
    description={A stopping heuristic, what is it?},
}

\newglossaryentry{glos:kl}{
    name={KL-Divergence},
    first={\emph{Kullback–Leibler Divergence}},
    description={What is KL-Divergence?}
}

\newglossaryentry{glos:stopping_strategy}{
    name={Stopping Strategy},
    plural={stopping strategies},
    description={A stopping strategy, what is it?},
}

\newglossaryentry{glos:precision}{
    name={Precision},
    first={precision},
    description={What is precision?},
}

\newglossaryentry{glos:serp}{
    name={SERP},
    first={SERP},
    description={What is a SERP?},
}

\newglossaryentry{glos:univac}{
    name={UNIVAC},
    first={UNIVAC},
    description={The \emph{Universal Automatic Computer}, one of the first electronic, stored-program computers utilised to execute what is considered to be a~\gls{glos:ir} system.},
}

\newglossaryentry{glos:html}{
    name={HTML},
    first={HTML},
    description={What is Hypertext Markup Language?},
}

\newglossaryentry{glos:rdbms}{
    name={RDBMS},
    first={RDBMS},
    description={What is a RDBMS?},
}

\newglossaryentry{glos:trec}{
    name={TREC},
    first={TREC},
    description={What is TREC?},
}

\newglossaryentry{glos:hci}{
    name={HCI},
    first={HCI},
    description={What is HCI?},
}

\newglossaryentry{glos:cg}{
    name={CG},
    first={CG},
    description={What is CG?},
}

\newglossaryentry{glos:dcg}{
    name={DCG},
    first={DCG},
    description={What is DCG?},
}

\newglossaryentry{glos:rbp}{
    name={RBP},
    first={RBP},
    description={What is RBP?},
}

\newglossaryentry{glos:ift}{
    name={IFT},
    first={IFT},
    description={What is IFT?},
}

\newglossaryentry{glos:csm}{
    name={CSM},
    first={CSM},
    description={What is the CSM?},
}

\newglossaryentry{glos:url}{
    name={URL},
    first={URL},
    description={What is a URL?},
}

\newglossaryentry{glos:http}{
    name={HTTP},
    first={HTTP},
    description={What is HTTP?},
}

\newglossaryentry{glos:hyperlink}{
    name={hyperlink},
    first={\emph{hyperlink}},
    firstplural={\emph{hyperlinks}},
    plural={hyperlinks},
    description={What is a hyperlink?}
}

\newglossaryentry{glos:internet}{
    name={Internet},
    first={\emph{Internet}},
    description={What is the internet?}
}

\newglossaryentry{glos:nist}{
    name={NIST},
    first={NIST},
    description={What is NIST?}
}

\newglossaryentry{glos:gpu}{
    name={GPU},
    first={GPU},
    description={What is a GPU?}
}

\newglossaryentry{glos:cpu}{
    name={CPU},
    first={CPU},
    description={What is a CPU?}
}