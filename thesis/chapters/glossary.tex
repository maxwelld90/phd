%!TEX TS-program = xelatex
%!TEX root = ../maxwell2018thesis.tex

% To refer to a glossary entry, use the \gls{} command.
% If there is an acronym for the glossary entry, refer to the acronym, which in turn cross-references the glossary entry.

% To add a new glossary entry, use the following template.
%
% \newglossaryentry{glos:NAME}{
%     name={GLOSSARY ENTRY},
%     description={DESCRIPTION},
%     first={\emph{FIRST MENTION}},
% }
%
% Note that first won't be used when the acronym is used.


\newglossaryentry{glos:ir}{
    name={IR},
    description={As a field of academic study, \emph{Information Retrieval} could be defined as the study of \emph{``finding material (usually documents) of an unstructured nature (usually text) that satisfies an information need from within large collections (usually stored on computers)''}~\citep{manning2008ir}.},
}

\newglossaryentry{glos:iir}{
    name={IIR},
    description={A simplistic description of \emph{Interactive Information Retrieval} would be the study of how humans interact with retrieval systems, considering aspects such as their behaviours and experiences. This is in contrast to the study of~\gls{acr:ir}, considering purely \emph{system-sided} aspects.},
}

\newglossaryentry{glos:www}{
    name={WWW},
    description={The \emph{World Wide Web} is an information space in which documents and other resources, linked together via~\glsplural{glos:hyperlink}, can be accessed via the~\gls{glos:internet}.},
}

\newglossaryentry{glos:kl}{
    name={KL-Divergence},
    first={\emph{Kullback–Leibler Divergence}},
    description={\emph{Kullback-Leibler} divergence, or \emph{relative entropy}, is a measure of how one probability distribution is different from a second. }
}

\newglossaryentry{glos:precision}{
    name={Precision},
    first={precision},
    description={What is precision?},
}

\newglossaryentry{glos:serp}{
    name={SERP},
    first={SERP},
    description={A Search Engine Results Page is the primary output of a contemporary retrieval system (typically~\gls{acr:www}-based). It is a page consisting of a series of results that were matched by the retrieval system to the searcher's \emph{query.}},
}

\newglossaryentry{glos:univac}{
    name={UNIVAC},
    first={UNIVAC},
    description={The \emph{Universal Automatic Computer}, one of the first electronic, stored-program computers utilised to execute what is considered to be a~\gls{glos:ir} system.},
}

\newglossaryentry{glos:html}{
    name={HTML},
    first={HTML},
    description={\emph{HyperText Markup Language} is the standard \emph{markup language} used in the development of web pages and web applications. HTML documents are annotated in a way that is syntactically different from the text, such as through the use of \texttt{<tags>}).},
}

\newglossaryentry{glos:rdbms}{
    name={RDBMS},
    first={RDBMS},
    description={A \emph{Relational DataBase Management System} is a type of database management system based upon the relational model devised by~\cite{codd1970rdbms}. At a minimum, a RDBMS provides data as a series of \emph{tables,} comprised of rows and columns, and the ability to create \emph{relationships} between said data.},
}

\newglossaryentry{glos:trec}{
    name={TREC},
    first={TREC},
    description={The \emph{Text REtrieval Conference} is an~\gls{acr:ir} evaluation forum, considering a number of different research areas, or \emph{tracks.} Central to the TREC effort is the development of topics, tasks and document collections that are commonly used in~\gls{acr:ir} experimentation -- with this thesis included.},
}

\newglossaryentry{glos:hci}{
    name={HCI},
    first={HCI},
    description={Human-Computer Interaction is the study of how computer technology is used and designed. It focuses upon the interfaces between users and computers.},
}

\newglossaryentry{glos:cg}{
    name={CG},
    first={CG},
    description={\emph{Cumulative Gain} is a measure of ranking quality, used to measure the effectiveness of a retrieval system (or the searchers that use it). The usefulness, or \emph{gain} possessed by a document is considered and accumulated to produce a final measure.},
}

\newglossaryentry{glos:dcg}{
    name={DCG},
    first={DCG},
    description={Considered as a natural evolution of~\gls{glos:cg}, \emph{Discounted Cumulative Gain} once again considers the \emph{gain} possessed by a document. However, the underlying assumption here is that relevant documents at higher ranks are more desirable. Gain therefore for documents at lower ranks are \emph{penalised,} or discounted, producing a rank-aware measure.},
}

\newglossaryentry{glos:rbp}{
    name={RBP},
    first={RBP},
    description={\emph{Rank-Biased Precision}~\citep{moffat2008rbp} is an evaluation measure used within~\gls{acr:ir}. It encodes within it a simple model of searcher behaviour, with a \emph{patience} factor denoting how far down a list of ranked results a searcher is prepared to go.},
}

\newglossaryentry{glos:ift}{
    name={IFT},
    first={IFT},
    description={\emph{Information Foraging Theory} applies the theory and constructs provided as part of~\gls{glos:oft}. It is based upon the assumption that when searching for information, individuals will employ instinctive foraging mechanisms that have evolved from helping animals find food in the wild. This was first considered in the 1990's, with seminal work published by~\cite{pirolli1999ift}.},
}

\newglossaryentry{glos:csm}{
    name={CSM},
    first={CSM},
    description={The \emph{Complex Searcher Model} is the high-level, conceptual searcher model proposed in this thesis. It considers the search session as a whole, and incorporates novel improvements to the search process, such as a new \emph{stopping decision point.}},
}

\newglossaryentry{glos:url}{
    name={URL},
    first={URL},
    description={A \emph{Uniform Resource Locator} is a reference to some computer resource hosted on a computer network. It contains the address to the resource, and the means by which it can be retrieved. For example, the URL \texttt{http://www.dmax.org.uk} specifies that~\gls{acr:http} is used to retrieve content at the address \texttt{www.dmax.org.uk}.},
}

\newglossaryentry{glos:fqdn}{
    name={FQDN},
    first={FQDN},
    description={A \emph{Fully Qualified Domain Name} is a domain that specifies a host's exact location within a domain name hierarchy. For example, \texttt{www} may be a valid hostname, but \texttt{www.gla.ac.uk} provides an exact match to the host's location within a wider network.},
}

\newglossaryentry{glos:http}{
    name={HTTP},
    first={HTTP},
    description={The \emph{HyperText Transfer Protocol} is the underlying protocol used for the transmission of content over the~\gls{glos:www}, amongst many other protocols. It defines the rules by which web servers and web browsers can communicate with one another.},
}

\newglossaryentry{glos:hyperlink}{
    name={hyperlink},
    first={\emph{hyperlink}},
    firstplural={\emph{hyperlinks}},
    plural={hyperlinks},
    description={A \emph{hyperlink} is a reference to some data source that can be clicked on to jump to said data source. This concept is most well known as part of the~\gls{glos:www}, with the links that hyperlinks create between documents defining the web-like structure.}
}

\newglossaryentry{glos:internet}{
    name={Internet},
    first={\emph{Internet}},
    description={The \emph{Internet} is a global system of interconnected computer networks. This \emph{network of networks} consists of private, public, business, governmental and academic computer networks all working together to provide \emph{routes} to computers, or hosts, all over the world.}
}

\newglossaryentry{glos:nist}{
    name={NIST},
    first={NIST},
    description={The \emph{National Institute for Standards and Technology} is a laboratory and non-regulatory agency of the \emph{U.S. Department of Commerce.} NIST has been central in providing support to the~\gls{acr:trec} evaluation effort.}
}

\newglossaryentry{glos:gpu}{
    name={GPU},
    first={GPU},
    description={What is a GPU?}
}

\newglossaryentry{glos:cpu}{
    name={CPU},
    first={CPU},
    description={What is a CPU?}
}

\newglossaryentry{glos:mturk}{
    name={MTurk},
    first={MTurk},
    description={\emph{Amazon Mechanical Turk} is a \emph{crowdsourcing} platform, allowing for one to co-ordinate the use of human intelligence to perform tasks that computers cannot presently do.}
}

\newglossaryentry{glos:set}{
    name={SET},
    first={SET},
    description={\emph{Search Economic Theory}~\citep{azzopardi2011economics} is a theory explaining the search process in terms of economics -- in particular \emph{microeconomic theory}. Under this approach, the search process is viewed as a series of \emph{inputs (queries, assessments)} that are used to produce an \emph{output (relevance).}}
}

\newglossaryentry{glos:esl}{
    name={ESL},
    first={ESL},
    description={The \emph{Expected Search Length} is an evaluation measure used within~\gls{acr:ir}. It considers the number of non-relevant documents that will have to be examined by a searcher before reaching a desired number of relevant documents. The ESL provides motivation for a number of~\glsplural{glos:stopping_heuristic} used within this thesis.}
}

\newglossaryentry{glos:oft}{
    name={OFT},
    first={OFT},
    description={\emph{Optimal Foraging Theory}~\citep{stephens1986foraging_theory} is a behavioural ecology model that helps to predict how animals behave when searching for food. An optimal foraging strategy is employed that provides the most gain (energy) at the lowest cost.}
}

\newglossaryentry{glos:mvt}{
    name={MVT},
    first={MVT},
    description={The \emph{Marginal Value Theorem}~\citep{charnov1976mvt} is an optimisation model used to describe the behaviour of an individual foraging in a system where resources are located in discrete \emph{patches.}}
}

% patch

\newglossaryentry{glos:stopping_heuristic}{
    name={Stopping Heuristic},
    first={stopping heuristic},
    plural={stopping heuristics},
    firstplural={stopping heuristics},
    description={A \emph{stopping heuristic} is defined in this thesis as a heuristic that describes the stopping behaviour of a searcher. A heuristic may consider one or more \emph{stopping criteria} when determining a stopping point.}
}

\newglossaryentry{glos:stopping_strategy}{
    name={Stopping Strategy},
    first={stopping strategy},
    plural={stopping strategies},
    firstplural={stopping strategies},
    description={A \emph{stopping strategy} is an operationalised~\gls{glos:stopping_heuristic}. The heuristic is converted to a series of rules that can be subsequently operationalised -- and later implemented as part of a wider searcher model.}
}

\newglossaryentry{glos:stopping_decision_point}{
    name={Stopping Decision Point},
    first={stopping decision point},
    plural={stopping decision points},
    description={We refer to \emph{stopping decision points} as the decisions within the~\gls{acr:csm} that permit a searcher to stop their current activity (i.e. examining result summaries, or the search session).}
}

\newglossaryentry{glos:prp}{
    name={PRP},
    first={PRP},
    description={The \emph{Probability Ranking Principle}~\citep{cooper1971relevance, robertson1977prp} is a fundamental theory of~\gls{acr:ir}, outlining that for a retrieval system to be effective, it must present results to a searcher in decreasing order of likelihood of the results being relevant.}
}

\newglossaryentry{glos:iprp}{
    name={iPRP},
    first={iPRP},
    description={The \emph{Interactive Probability Ranking Principle}~\citep{fuhr2008iprp} is an update to the~\gls{glos:prp}. It considers interaction within its framework, allowing for costs over different activities (i.e. issuing queries, examining documents) and changes in a searcher's information need.}
}