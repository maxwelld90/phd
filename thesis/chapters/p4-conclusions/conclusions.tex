%!TEX TS-program = xelatex
%!TEX root = ../../maxwell2018thesis.tex

\chapter{Conclusions and Future Work}
\section{Discussion and Contributions}

\subsection{User Modelling}
\subsection{Examining Stopping Behaviours}

\section{Conclusions}

\section{Future Research Directions}


When considering possible future directions, Apple’s 1987 Knowledge Navigator vision of IR is still a strong exemplar of how search systems might develop. The short film showed a college professor pulling together a lecture presentation at the last minute. The professor used a form of tablet computer running an IR system presented as an agent capable of impeccable speech recognition, natural dialogue management, a high level of semantic understanding of the searcher’s information needs, as well as unbounded access to documents and federated databases.
The Knowledge Navigator identified and connected the professor to a colleague who helped him with the lecture. The broader implications of finding people (rather than documents) to aid with information needs that we see facilitated in the vast growth of social media was not really addressed in the Apple vision. What it also did not encompass was the portability of computer devices opening the possibility of serving information needs pertinent to the particular local context of location, location type, route, the company one is in, or a combination of all these factors. \todo{from the history of IR paper}