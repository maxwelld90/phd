%!TEX TS-program = xelatex
%!TEX root = ../../maxwell2018thesis.tex

\chapter[Discussion, Conclusions and Future Work]{Discussion, Conclusions\\and Future Work}\label{chap:conclusions}
With all experimental work now reported, this final chapter provides a conclusion for this thesis. In particular, we provide a high-level summary of the thesis and the reported results, as well a discussion of the results from our simulated analyses. In particular, we emphasise the impact of these findings on~\gls{acr:ir} and~\gls{acr:iir} research. We then outline several potential future research directions, before concluding with some final remarks.

\section{Thesis Summary}
in a very short space, what did we examine?

- a table of results?
- a table of the conditions/interfaces?

\section{Discussion}
so there is the summary.
however, there's a lot of questions about why we got the results that we did.
why is that so?
in this section, we discuss a number of our findings, exploring why the results are what they are.

\subsection{Improving Realism}

\subsection{Stopping Strategies and Behaviours}

- simple strategies always perform best.
- the more complex, the worse the performance and approximations get.
    - this could be because of the way we instantiate.
    - IFT/difference.
        - you would expect perhaps with difference, the only strategies that look at the terms, that improvements would be observable as snippet lengths increase. however, this is not the case.

- combination strategies do well!
    - people consider how annoyed/satisfied they are. not a single criterion for judging when to stop.
    -

\subsection{Running Simulations of Interaction}

- then move onto a discussion of our simulation findings.
- talk about the different issues that we raised.

- moving to combination approaches, like bejewled~\cite{zhang2017bejewled}, etc.

- talk about the simulations
    - running the simulations. expensive. complicated.
    - as we introduced the SERP level stopping decision point, the complexity increased massively -- and so did the time required to run them.
    - talk about the pre-rolled judgements, too. methodological contribution. ensures repeatable, reproducible research.

- did we do enough runs?
    - we could justify why 50 DM runs was chosen.
    - but was this enough? may not be.
    - maybe we dont have enough trials.
    - and when we dont have enough trials to average over, the power of our experiments could be insufficient.
        - so we could be missing trends. we could be missing things that we simply cannot see.
        - limitation placed upon me by the hardware available to run the experiments. future work.


\section{Conclusion}

- what is the conclusion?
    - there is no overall winning combination. simple strategies seem to do better in terms of offering outright performance, and approximations. in terms of approximations, tuning a given stopping strategy will yield good approximations, regardless of the strategy you follow. evident with SS1-FIX for example, with comparatively deep depths (i.e. x1=24).

- we also see improvements in modelling.
- at the expense of increased complexity (in terms of running them), the simulations have been shown to be more realistic in terms of approximating click depths.
    - it does appear that taking scent into account is important. leads to slightly lower error rates, although we could not demonstrate this difference to be significant.

- limitations of the work could influence work, hide trends.
- nevertheless, it is an important contribution that demonstrates how task and interface affect stopping behaviour.

\section{Future Research Directions}

Additional Stopping Decision point -- improving realism further

Additional search contexts

additional stopping heuristics

encoding additional stopping models and measures
    - turning SS5/SS11 into a stopping model. e.g. bejewled, following on that train of thought.

existing stopping strategy operationalisation

heuristic selection

individual searcher stopping

scaling experiments up further.
    power idea in discussion.


\section{Final Remarks}

This is complex. Contributions

Now that this is all done, \emph{I} am off to the pub.