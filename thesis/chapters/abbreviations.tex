%!TEX TS-program = xelatex
%!TEX root = ../maxwell2018thesis.tex

% Ensure that \makeglossaries is called before anything else is done regarding glossaries/abbreviations.
\glsSetCompositor{-}
\makeglossaries

% Abbreviations, separate from glossary entries.
%
% If you want a new abbreviation referring to a glossary entry, use the following template.
%
% \newglossaryentry{acr:ACRONYM}{
%     type=\acronymtype,
%     name={ACRONYM},
%     first={\emph{SPELL IT OUT}\glsadd{glos:ACRONYM}},
%     description={SPELL IT OUT\glossarysee{\gls{glos:ACRONYM}}},
% }
%
% If you want an abbreviation on its own, use the following template.
%
% \newglossaryentry{acr:ACRONYM}{
%     type=\acronymtype,
%     name={ACRONYM},
%     first={\emph{SPELL IT OUT}},
%     description={SPELL IT OUT WITH A SHORT DESCRIPTION},
% }


\newglossaryentry{acr:ir}{
    type=\acronymtype,
    name={IR},
    first={\emph{Information Retrieval (IR)}\glsadd{glos:ir}},
    description={Information Retrieval\glossarysee{\gls{glos:ir}}},
}

\newglossaryentry{acr:iir}{
    type=\acronymtype,
    name={IIR},
    first={\emph{Interactive Information Retrieval (IIR)}\glsadd{glos:iir}},
    description={Interactive Information Retrieval\glossarysee{\gls{glos:iir}}},
}

\newglossaryentry{acr:www}{
    type=\acronymtype,
    name={WWW},
    first={\emph{World Wide Web (WWW)}\glsadd{glos:www}},
    description={World Wide Web\glossarysee{\gls{glos:www}}},
}