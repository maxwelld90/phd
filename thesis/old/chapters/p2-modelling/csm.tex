%!TEX TS-program = xelatex
%!TEX root = ../../maxwell2018thesis.tex

\chapter[Advancing User Modelling in IIR]{Advancing User Modelling in IIR}

``Empirical studies of complete systems mostly focus on variations of single components''~\citep{fuhr2008iprp}

In previous chapter we have seen a number of different entire search session models.
In this chapter, we introduce a number of models that have been considered throughout this PhD, with a focus on stopping decision points.

- Why do we consider this as a flow chart?
It's a high-level, conceptual model, where different states/activities are represented as components.
Totally up to how one implements each component -- i.e. with stopping components, we can instantiate them in a number of different ways.

- these models are based on stochastic approaches.


\section{The Basic User Model}
Query, Document, Mark

\section{The Complex Searcher Model}
introduce the model, explaining and motivating it

\subsection{CSM, Mark I}
Query, Snippet, Document, Mark

\subsection{CSM, Mark II}
- motivation for including the SERP
- Query, SERP, Snippet, Document, Mark
- ECIR 2018 paper

\subsection{Considering State and Agency}
- as an aside, we can also incorporate state and agency into the model.
- to a degree, the mark I and mark II models already do consider some form of state, by virtue of having a record of what documents have been previously examined, for example.
- however, we can go further here, introducing more advanced state, leading to some form of agency. for example, an agent can determine a new set of queries based upon what it has previously observed.

- while not central to the thesis, we can at least consider it, and demonstrate that it is indeed useful.


\section{Evaluating Model Effectiveness}
Core to developing a new model is to basically improve our representation of a real world phenomenon, such as, in this case, the search process.

\subsection{Research Questions}
Allowing us to address HL-RQ1.

\subsection{Experimental Design}

\subsection{Results}

% \section{Introduction}
% \section{Components of the Model}
% \section{Evolution of the Model}
% \section{Incorporating State and Agency}
% \subsection{Section Introduction}
% \subsection{Research Questions}
% \subsection{Experimental Design}
% query reformulation is a good contribution here. the agent is able to generate queries based upon what it has seen.
% \subsection{Results}
% \subsection{Discussion and Conclusions}
% \section{Chapter Summary}