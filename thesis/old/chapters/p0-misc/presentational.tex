%!TEX TS-program = xelatex
%!TEX root = ../../maxwell2018thesis.tex

\begin{preamble}[Presentational Conventions]
\phantomsection
\addcontentsline{toc}{part}{Presentational Conventions}

A number of different presentational conventions have been employed in this thesis. These are listed below for reference.

\begin{itemize}
    
    \item{Spelling is according to the \emph{Oxford English Dictionary}; the version referred to is searchable online at \url{https://en.oxforddictionaries.com/}.}
    
    \item{\emph{Italicised text} is used to define a term, but not thereafter. This applies to acronyms, where the full expansion is presented initially; associated abbreviations are used thereafter.}
    
    \item{Research questions and other key shorthand descriptions for components of this research are presented inline within a \bluebox{shaded} box.}
    
    \item{Pseudo-code that is presented within this thesis uses the \emph{HAGGIS} high-level reference programming language, as per~\cite{cutts2014haggis}. This is particularly relevant for a Scottish PhD!}
    
\end{itemize}

This thesis is typeset in 12-point Palatino (body) and Foundry Sterling (headers) using \emph{XeTeX} version 0.99998, using a custom style complying with University of Glasgow thesis regulations.
\end{preamble}